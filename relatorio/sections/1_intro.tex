\begin{multicols}{2}
    \tableofcontents
    \section{Introdução}
    
O presente relatório delineia uma experiência de aprendizado de máquina em que comparamos o desempenho de um modelo de Rede Neural construído com a biblioteca PyTorch com um algoritmo k-Nearest Neighbors (KNN) do Scikit-Learn. Ambos os modelos foram treinados e testados no conhecido dataset Iris, um conjunto de dados padrão na área de aprendizado de máquina.

O dataset Iris contém 150 amostras de três espécies de flores de íris (setosa, versicolor, virginica), cada uma com 50 amostras. Cada amostra possui quatro características distintas: comprimento da sépala, largura da sépala, comprimento da pétala e largura da pétala. O objetivo do experimento foi treinar ambos os modelos para prever corretamente a espécie da flor de íris a partir dessas quatro características.

O experimento foi dividido em duas partes principais. Na primeira parte, utilizamos a biblioteca PyTorch para construir e treinar uma Rede Neural. O modelo foi estruturado com uma camada de entrada correspondente às quatro características das flores, duas camadas ocultas e uma camada de saída correspondente às três espécies de íris.

Na segunda parte, utilizamos o algoritmo KNN usando a biblioteca Scikit-Learn. Esse algoritmo classifica uma amostra com base na maioria das classes de seus vizinhos mais próximos.

O intuito deste relatório é apresentar os resultados obtidos, comparar o desempenho de ambos os modelos e discutir as vantagens e desvantagens de cada método para a tarefa de classificação proposta.
    \end{multicols}