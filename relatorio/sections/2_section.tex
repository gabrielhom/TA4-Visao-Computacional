{\color{gray}\hrule}
\begin{center}
\section{O Dataset Iris}
\end{center}
{\color{gray}\hrule}
\begin{multicols}{2}

    O dataset Iris é um dos datasets mais famosos e acessíveis na área de aprendizado de máquina. Foi introduzido pelo estatístico e biólogo britânico Ronald Fisher em 1936 em seu artigo "The use of multiple measurements in taxonomic problems" como um exemplo de análise discriminante linear.
    
    O dataset Iris é composto por 150 observações de íris, uma flor que tem variedades distintas. As amostras pertencem a uma de três espécies de íris: Iris setosa, Iris versicolor e Iris virginica. Cada espécie é representada por 50 observações, tornando o conjunto de dados balanceado.
    
    Cada observação no conjunto de dados consiste em quatro medidas (em cm) dessas flores: 
    
    1. Comprimento da sépala
    
    2. Largura da sépala

    3. Comprimento da pétala

    4. Largura da pétala
    
    Essas quatro características constituem a base do aprendizado de máquina para este conjunto de dados. A tarefa consiste em prever a espécie da íris com base nessas medidas.
    
    O dataset Iris é adequado para classificação e clusterização. Em termos de classificação, a tarefa é geralmente definida como um problema de classificação multiclasse, onde o objetivo é prever uma das três espécies possíveis. 
    
    Este conjunto de dados tem sido extremamente influente, pois é pequeno, não possui valores faltantes, e exige pouca preparação. Além disso, apesar de sua simplicidade, pode-se explorar diferentes técnicas de aprendizado de máquina e pré-processamento. Dada a sua natureza multiclasse, é ideal para técnicas de classificação e avaliação, e também é adequado para visualização de dados e técnicas de redução de dimensionalidade.
    
    No experimento, utilizamos o dataset Iris para treinar e testar um modelo de Rede Neural usando a biblioteca PyTorch e um algoritmo KNN do Scikit-Learn, comparando posteriormente o desempenho de ambos os modelos.
\end{multicols}